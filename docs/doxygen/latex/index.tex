Python is a (confusingly named) Monte Carlo radiative transfer code which uses the Sobolev approximation. It has been developed by Knox Long, Christian Knigge, Stuart Sim, Nick Higginbottom, James Matthews and Sam Mangham. The code has been used for a variety of research projects invovling the winds of cataclysmic variables, of young stellar objects, and of A\+GN.

The code is under active development, but we are looking for beta users to test the code, and potentially use it for their own research. If you are interested in using Python please email the collaboration at agn-\/winds\mbox{[}at\mbox{]}googlegroups\mbox{[}dot\mbox{]}com or contact Knox Long via long\mbox{[}at\mbox{]}stsci\mbox{[}dot\mbox{]}edu.

Documentation of input parameters is hosted on \href{http://agnwinds.readthedocs.io/en/dev/}{\tt Read\+The\+Docs}.

\subsection*{Travis Build Status}

\href{https://travis-ci.org/agnwinds/python}{\tt }

\subsection*{Installation}

Python and the various routines associated are set up in a self-\/contained directory structure. The basic directory structure and the data files that one needs to run Python need to be retrieved and compiled.

If you want to obtain a stable (!) release, go to the \href{https://github.com/agnwinds/python/releases}{\tt Releases} page.

If you want to download the latest dev version, you can zip up the git repository by clicking on the zip icon to the right of the Git\+Hub page. Aternatively, you can clone the repository using \begin{DoxyVerb}$ git clone https://github.com/agnwinds/python.git 
\end{DoxyVerb}


If you anticipate contributing to development we suggest Forking the repository and submitting pull requests with any proposed changes.

Once you have the files, you need to cd to the new directory and set your environment variables \begin{DoxyVerb}$ export PYTHON = /path/to/python/
$ cd $PYTHON 
$ ./configure
$ make install
$ make clean
\end{DoxyVerb}


note that export syntax is for bash-\/ for csh use \begin{DoxyVerb}$ setenv PYTHON /path/to/python/
\end{DoxyVerb}


Atomic data is stored in our \href{https://github.com/agnwinds/data}{\tt data repository} with it\textquotesingle{}s own releases page. one should unzip these files and place them in a \$\+P\+Y\+T\+H\+ON/data folder.

A development user may want to work on atomic data as part of their work, and pull in changes as they are made, in which case we recommend cloning the data repository\+: \begin{DoxyVerb}$ cd $PYTHON; git clone https://github.com/agnwinds/data data
\end{DoxyVerb}


\subsection*{Running python}

To run python you need to add the following to your \$\+P\+A\+TH variable\+: \begin{DoxyVerb}$PYTHON/bin
\end{DoxyVerb}


You can then setup your symbolic links by running \begin{DoxyVerb}$ Setup_Py_Dir
\end{DoxyVerb}


and run the code by typing, e.\+g. \begin{DoxyVerb}$ py root.pf
\end{DoxyVerb}


Please see the \href{https://github.com/agnwinds/python/wiki/Installing-and-Running-Python}{\tt wiki} and docs folder for how to use the code.

Any comments, email \href{mailto:jm8g08@soton.ac.uk}{\tt jm8g08@soton.\+ac.\+uk} or \href{mailto:long@stsci.edu}{\tt long@stsci.\+edu}. 